\documentclass[a4paper,9pt]{scrartcl}
\usepackage[english]{babel}
\usepackage[utf8]{inputenc}
\usepackage{amssymb,amsmath}
\usepackage{graphicx}

\DeclareMathOperator{\sech}{sech}
\DeclareMathOperator{\csch}{csch}
\DeclareMathOperator{\cosech}{cosech}
\DeclareMathOperator{\arsec}{arcsec}
\DeclareMathOperator{\arcot}{arcot}
\DeclareMathOperator{\arcsc}{arcsc}
\DeclareMathOperator{\arcosh}{arcosh}
\DeclareMathOperator{\arsinh}{arsinh}
\DeclareMathOperator{\artanh}{artanh}
\DeclareMathOperator{\arsech}{arsech}
\DeclareMathOperator{\arcsch}{arcsch}
\DeclareMathOperator{\arcoth}{arcoth}

\usepackage{geometry}
\geometry{a4paper,left=18mm,right=18mm, top=2cm, bottom=2cm}
\usepackage{color}
\title{Further Maths}

\usepackage{multirow}
\usepackage{lipsum}
\usepackage{ctex}
\usepackage{enumerate}

\usepackage{pgfplots}
\usepgfplotslibrary{polar,colormaps}
\usepackage{mathpazo}
\usepackage{tikz}
\usepackage{gauss}
\usepackage{xcolor}
\usepackage{blindtext}

\usetikzlibrary{datavisualization.polar}

\pgfplotsset{compat=newest}

% define the plot style and the axis style
\tikzset{elegant/.style={smooth,thick,samples=50,magenta}}

\pgfplotsset{compat=1.8}

%\usepackage{geometry}
%\usepackage[utf8]{inputenc}
%\usepackage[T1]{fontenc}
%\usepackage{fontspec}
%\setmainfont{JetBrains Mono}

\newcommand{\vecabs}[1]{\left| \vec{#1} \right|}
\newcommand{\abs}[1]{\left| #1 \right|}

\begin{document}
    \section{Binomial distribution}

    \begin{math}
        X \sim \left( n, p \right)
    \end{math}

    \subsection{Conditions for binomial distribution}

    \begin{itemize}
        \item n repeated trials
        \item independent trials
        \item two outcomes
        \item P is a constant
    \end{itemize}


    \section{Poisson distribution}

    \subsection{Conditions}
    \begin{itemize}
        \item single in space or time
        \item independent of each other
        \item at a constant rate
    \end{itemize}


    \section{PDF and CDF}

    PDF is $f(x)$, CDF is $F(x)$

    \subsection{mode}

    Mode is the x value at which the PDF function has the greatest probability.

    \subsection{mean}

    \textit{Use formulae provided for PDF.}

    \subsection{median}

    \begin{math}
        F(median)\,=\,0.5
    \end{math}


    \section{Skewness}

    \begin{tabular}{|c|c|}
        \hline symmetrical   & $mode = median = mean$ \\
        \hline positive skew & $mode < median < mean$ \\
        \hline negative skew & $mode > median > mean$ \\
        \hline
    \end{tabular}


    \section{Approximation}

    \subsection{Binomial to Poisson}

    \begin{math}
        X {\sim} B(n,p) \longrightarrow X {\sim} Po(np)
    \end{math}

    \subsection{Binomial to Normal}

    \begin{math}
        X {\sim} B(n,p) \longrightarrow X {\sim} N(np, npq)
    \end{math}

    \subsubsection{Conditions}

    \begin{math}
        \left\{
        \begin{matrix}
            np &>& 5 \\
            nq &>& 5
        \end{matrix}
        \right.
    \end{math}

    \subsection{Poisson to Normal}

    \begin{math}
        X {\sim} Po(\lambda) \longrightarrow X {\sim} N(\lambda, \lambda)
    \end{math}

    \subsubsection{Conditions}

    \begin{math}
        \lambda > 10
    \end{math}


    \section{Continuous uniform distribution}

    \begin{math}
        X {\sim} $U[a, b]$
    \end{math}

    \subsection{CDF for U}

    \begin{math}
        F(x) = \left\{ \begin{matrix}
                           a, & x<a, \\
                           \frac{x-a}{b-a}\,, & a \leq x \leq b,\\
                           1, & x>b.
        \end{matrix}\right.
    \end{math}


    \section{Sampling}

    \subsection{Terms}

    \subsubsection{population}

    a collection of individual items.

    \subsubsection{sample}

    a selection of individual members or items from a population.

    \subsubsection{finite population}

    each individual member can be given a number.

    \subsubsection{infinite population}

    impossible to number each member.

    \subsubsection{sampling unit}

    an individual member of a population.

    \subsubsection{sampling frame}
    a list of sampling units used in practice to represent a population.

    \subsubsection{statistic}

    a quantity calculated solely from the observations in a sample.

    \subsubsection{sampling distribution}

    defined by giving all possible values of the statistic and the probability of each occurring.


    \section{Hypothesis testing}

    \subsection{Terms}

    \subsubsection{hypothesis test}

    \textbf{Hypothesis test} is a mathematical procedure to examine value of a population parameter proposed by the null hypothesis $H_0$, compared to the alternative hypothesis $H_1$.

    \subsubsection{test statistic}

    In a hypothesis test the evidence comes from a sample which is summarised in the form of a \textbf{test statistic}.

    \subsubsection{critical region}

    The range of values of a test statistic that would lead you to reject $H_0$.

    \subsubsection{critical value}

    The boundary value of a critical region.

    \subsubsection{one-tailed test}

    Looks either for an increase or for a decrease in a parameter, and has a single critical value.

    \subsubsection{Two-tailed test}

    Looks for both an increase and a decrease in a parameter, and has two critical values.

    \subsubsection{actual significance level}

    The probability of rejecting $H_0$.

\end{document}