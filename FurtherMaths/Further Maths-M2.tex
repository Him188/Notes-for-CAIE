\documentclass[a4paper,9pt]{scrartcl}
\usepackage[english]{babel}
\usepackage[utf8]{inputenc}
\usepackage{amssymb,amsmath}
\usepackage{graphicx}

\DeclareMathOperator{\sech}{sech}
\DeclareMathOperator{\csch}{csch}
\DeclareMathOperator{\cosech}{cosech}
\DeclareMathOperator{\arsec}{arcsec}
\DeclareMathOperator{\arcot}{arcot}
\DeclareMathOperator{\arcsc}{arcsc}
\DeclareMathOperator{\arcosh}{arcosh}
\DeclareMathOperator{\arsinh}{arsinh}
\DeclareMathOperator{\artanh}{artanh}
\DeclareMathOperator{\arsech}{arsech}
\DeclareMathOperator{\arcsch}{arcsch}
\DeclareMathOperator{\arcoth}{arcoth}

\usepackage{geometry}
\geometry{a4paper,left=18mm,right=18mm, top=2cm, bottom=2cm}
\usepackage{color}
\title{Further Maths}

\usepackage{multirow}
\usepackage{lipsum}
\usepackage{ctex}
\usepackage{enumerate}

\usepackage{pgfplots}
\usepgfplotslibrary{polar,colormaps}
\usepackage{mathpazo}
\usepackage{tikz}
\usepackage{gauss}
\usepackage{xcolor}

\usetikzlibrary{datavisualization.polar}

\pgfplotsset{compat=newest}

% define the plot style and the axis style
\tikzset{elegant/.style={smooth,thick,samples=50,magenta}}

\pgfplotsset{compat=1.8}

%\usepackage{geometry}
%\usepackage[utf8]{inputenc}
%\usepackage[T1]{fontenc}
%\usepackage{fontspec}
%\setmainfont{JetBrains Mono}

\newcommand{\vecabs}[1]{\left| \vec{#1} \right|}
\newcommand{\abs}[1]{\left| #1 \right|}

\begin{document}
    \section{Centre of mass}

    \subsection{General centre of mass}

    Shape $A(x_1, y_1)$ weighted $w_1$, $B(x_2, y_2)$ weighted $w_2$, $C(x_3, y_3)$ weighted $w_3$, centre of mass $G(\overline{x}, \overline{y})$:

    \begin{displaymath}
        w_1\begin{pmatrix}
               x_1 \\
               y_1
        \end{pmatrix} +w_2\begin{pmatrix}
                              x_2 \\
                              y_2
        \end{pmatrix}+w_3\begin{pmatrix}
                             x_3 \\
                             y_3
        \end{pmatrix}=(w_1+w_2+w_3)\begin{pmatrix}
                                       \overline{x}\\
                                       \overline{y}
        \end{pmatrix}
    \end{displaymath}

    \subsection{Centre of mass of uniform triangular lamina}

    \begin{displaymath}
        G\left( \frac{x_1+x_2+x_3}{3}, \frac{y_1+y_2+y_3}{3} \right)
    \end{displaymath}

    \subsection{Centre of mass of uniform sector of a circle}

    Make the origin at the centre of the circle, make x-axis crossing the symmetry of the sector, then

    \begin{displaymath}
        distance = \frac{2r\sin\alpha}{3\alpha}
    \end{displaymath}

    \begin{displaymath}
        G\left( 0, \frac{2r\sin\alpha}{3\alpha} \right)
    \end{displaymath}

    \subsection{Combined centre of mass of two laminae}
    Shape $A$, $G_A(x_a, y_a)$ area $A_a$; \\
    Shape $B$, $G_B(x_b, y_b)$ area $A_b$. \\
    Combined centre of mass:

    \begin{displaymath}
        A_a\begin{pmatrix}
               x_a \\
               y_a
        \end{pmatrix} +A_b\begin{pmatrix}
                              x_b \\
                              y_b
        \end{pmatrix}=(A_a+A_b)\begin{pmatrix}
                                   \overline{x}\\
                                   \overline{y}
        \end{pmatrix}
    \end{displaymath}

    \subsection{Centre of mass of an arc of a circle}

    Make the origin at the centre of the circle, make x-axis crossing the symmetry of the sector, then

    \begin{displaymath}
        distance = \frac{r\sin\alpha}{\alpha}
    \end{displaymath}

    \subsection{Centre of mass of a frame}

    Consider all lines as a dedicated shape, use the general method to calculate its centre then use centre point to represent the shape then apply the general method again.


    \section{Further Center of Mass}

    \subsection{Center of mass of a lamina}

    \begin{displaymath}
        \overline{x} = \frac{\int_{a}^{b} xy \,dx}{\int_{a}^{b} y \,dx}
    \end{displaymath}

    \begin{displaymath}
        \overline{y} = \frac{\int_{a}^{b} \frac{1}{2}y^2 \,dx}{\int_{a}^{b} y \,dx}
    \end{displaymath}

    Or calculate separately,

    \begin{displaymath}
        M = \int_{a}^{b} {\rho}y \,dx
    \end{displaymath}

    \begin{displaymath}
        M\,\overline{x} = \int_{a}^{b} {\rho}xy \,dx
    \end{displaymath}

    \begin{displaymath}
        M\,\overline{y} = \int_{a}^{b} \frac{1}{2} {\rho}y^2 \,dx
    \end{displaymath}

    \subsection{Center of mass of a solid mass rotated about x-axis}

    \begin{displaymath}
        \bar{x} = \frac{\int_{a}^{b} {\pi} y^2 x\,dx}{\int_{a}^{b} \pi y^2 \,dx}
    \end{displaymath}

    \begin{displaymath}
        M = \int_{a}^{b} \rho {\pi} y^2 \,dx
    \end{displaymath}

    \begin{displaymath}
        M\bar{x} = \int_{a}^{b} \rho {\pi} y^2 x\,dx
    \end{displaymath}

    \subsection{Center of mass of a solid mass rotated about y-axis}

    \begin{displaymath}
        \bar{y} = \frac{\int_{a}^{b} {\pi} x^2 y\,dx}{\int_{a}^{b} \pi x^2 \,dx}
    \end{displaymath}

    \subsection{Center of mass of solid hemisphere}

    \begin{displaymath}
        d = \frac{3}{8}\,r
    \end{displaymath}

    \subsection{Center of mass of hemisphere shell}

    \begin{displaymath}
        d = \frac{1}{2}\,r
    \end{displaymath}

    \subsection{Center of mass of circular cone}

    \begin{displaymath}
        d = \frac{1}{4}\,h
    \end{displaymath}

    \subsection{Center of mass of hollow cone}

    \begin{displaymath}
        d = \frac{1}{3}\,h
    \end{displaymath}


    \section{Momentum}

    \subsection{Coefficients of restitution}

    \begin{displaymath}
        e = \frac{v_2-v_1}{u_1-u_2}
    \end{displaymath}

    When $e=1$, \textit{perfect elastic}

    \subsection{Collisions with wall}

    \begin{displaymath}
        e = \frac{v}{u}
    \end{displaymath}


    \section{Stiffness coefficient}

    \subsection{Force by extended string}
    \begin{displaymath}
        F = \frac{{\lambda}x}{L}
    \end{displaymath}

    \subsection{Energy stored in string}
    \begin{displaymath}
        W = \frac{{\lambda}x^2}{2L}
    \end{displaymath}
\end{document}