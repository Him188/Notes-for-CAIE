\documentclass[a4paper,9pt]{scrartcl}
\usepackage[english]{babel}
\usepackage[utf8]{inputenc}
\usepackage{amssymb,amsmath}
\usepackage{geometry}
\geometry{a4paper,left=18mm,right=18mm, top=2cm, bottom=2cm}

\title{Further Maths}

\usepackage{multirow}
\usepackage{lipsum}
\usepackage{ctex}
\usepackage{enumerate}
%\usepackage{geometry}
%\usepackage[utf8]{inputenc}
%\usepackage[T1]{fontenc}
%\usepackage{fontspec}
%\setmainfont{JetBrains Mono}

\begin{document}
    \section{Summation of Series}\label{sec:summation-of-series}

    \begin{alignat*}{2}
        &\sum_{x=0}^{n}x    &= \frac{n(n+1)}{2} \\
        &\sum_{x=0}^{n}x^2  &= \frac{n(n+1)(2n+1)}{6} \\
        &\sum_{x=0}^{n}x^3  &= \frac{n^2(n+1)^2}{4}
    \end{alignat*}


    \section{Complex Numbers}\label{sec:complex-numbers}

    \begin{itemize}
        \item [1)] Translation

        $w=z+a+bi$\ : translation by
        $\begin{pmatrix}
             a \\b
        \end{pmatrix}$

        \item [2)] Enlargement

        $w=kz$\ : enlargement by a scale factor k

        \item [3)] Enlargement followed by translation

        $w=kz+a+bi$\ : enlargement by a scale factor k followed by a translation by
        $\begin{pmatrix}
             a \\b
        \end{pmatrix}$
    \end{itemize}

    \subsection{Transformations}

    \subsubsection{Example 1}
    Find the transformation $w = \frac{1}{z}, z != 0$, find the locus of $w$ when $z$ lies on the line with equation $y = 2x + 1$

    \begin{displaymath}
        x + yi = \frac{1}{u + vi} = \frac{u - vi}{u^2 + v^2} = \frac{u}{u^2+v^2} + \frac{-v}{u^2+v^2}i
    \end{displaymath}


    \section{Differentiation}

    \subsection{First order differentiation}

    \begin{displaymath}
        f(x) \frac{dy}{dx} + f'(x) y = \frac{d(f(x) y )}{dx}
    \end{displaymath}

    \textbf{Integration factor}: $\boxed{e^{\int{p}dx}}$

    \begin{displaymath}
        \frac{dy}{dx}+py=Q \rightrightarrows \frac{d(\boxed{e^{\int{p}dx}}y)}{dx} = \boxed{e^{\int{p}dx}} Q
    \end{displaymath}

    \subsection{Second order differentiation}

    \begin{displaymath}
        a\frac{d^{2}y}{dx^2} + b\frac{dy}{dx} + cy = 0
    \end{displaymath}

    \subsubsection{Auxiliary equation}
    \begin{displaymath}
        am^2 + bm + c = 0
    \end{displaymath}

    \textbf{If $\Delta > 0$, it has two distinct roots $\alpha$, $\beta$.}
    General solution:
    \begin{displaymath}
        y = Ae^{{\alpha}x} + Be^{{\beta}x}
    \end{displaymath}

    \textbf{If $\Delta = 0$, it has two repeated roots.}
    General solution:
    \begin{displaymath}
        y = (A+Bx)e^{{\alpha}x}
    \end{displaymath}

    \textbf{If $\Delta < 0$, it has two complex roots, $p + qi$ and $p - qi$. }
    General solution:
    \begin{displaymath}
        y = e^{px}(A\cos{qx}+B\sin{qx})
    \end{displaymath}

    \subsubsection{Example for finding a general solution for Second order differentiation}

    \begin{displaymath}
        \frac{d^{2}y}{dx^2} + 5\frac{dy}{dx} + 6y = 0
    \end{displaymath}
    \begin{displaymath}
        a = 1, b = 5, c = 6
    \end{displaymath}
    \begin{displaymath}
        m^2 + 5m + 6 = 0
    \end{displaymath}
    \begin{displaymath}
        m = -2 or m = -3
    \end{displaymath}
    \begin{displaymath}
        y = Ae^{-2x}+Be^{-3x}
    \end{displaymath}

    \subsubsection{Complementary functions}

    \begin{displaymath}
        a\frac{d^{2}y}{dx^2} + b\frac{dy}{dx} + cy = f(x)
    \end{displaymath}

    Solution: $y = complementary function + particular integral$ \\

    Particular integral is the general form of $f(x)$. \\

    \subsubsection{Complementary functions example}

    \begin{displaymath}
        \frac{d^{2}y}{dx^2} -8\frac{dy}{dx} + 12y = 36x
    \end{displaymath} \\
    \begin{itemize}
        \item [Step 1.] State CF and PI\\
        CF: $y = Ae^{2x}+Be^{6x}$\\
        PI: $y = {\lambda}x + \mu$\\

        \item [Step 2.] Differentiate PI\\
        Obtain: \\
        \begin{displaymath}
            \frac{dy}{dx} = \lambda
        \end{displaymath}
        \begin{displaymath}
            \frac{d^{2}y}{dx^2} = 0
        \end{displaymath}

        \item [Step 3.] Substitute $\frac{d^{2}y}{dx^2}$, $\frac{dy}{dx}$, $y$ into the differentiation equation.\\
        Then find $\lambda$ and $\mu$.
    \end{itemize}

    \subsection{Appendix: Particular Integrals}
    \begin{tabular}{|c|c|c|}
        \hline $f(x)$         & Particular integral                                  \\
        \hline $k$            & $\lambda$                                            \\
        \hline $ax+b$         & ${\lambda}x+\mu$                                     \\
        \hline $ax^2+bx+c$    & ${\lambda}x^2+{\mu}x+\gamma$                         \\
        \hline $ae^{kx}$      & ${\lambda}e^{kx}$                                    \\
        \hline $a\sin{kx}$    & \multirow{3}{*}{${\lambda}\sin{kx}+{\mu}{\cos{kx}}$} \\
        $a\sin{kx}$           &                                                      \\
        $a\sin{kx}+b\cos{kx}$ &                                                      \\
        \hline
    \end{tabular}

    \subsection{Appendix: Formulas of Integration and Differentiation}

    \begin{displaymath}
        \int \frac{f'(x)}{f(x)} dx = \ln(f(x))
    \end{displaymath}

    \begin{displaymath}
        \int{\tan{x}\sin{x}}dx = \sec{x} + C
    \end{displaymath}

    \begin{displaymath}
        \int{\cot{x}\csc{x}}dx = -\csc{x} + C
    \end{displaymath}

    \begin{displaymath}
        \int{\sec{x}}dx = \ln{(\sec{x} + \tan{x})} + C
    \end{displaymath}

    \begin{displaymath}
        \int{\csc{x}}dx = -(\ln{\csc{x} + \cot{x}}) + C
    \end{displaymath}
    \begin{displaymath}
        \int{\csc{x}}dx = -(\ln{\csc{x} + \cot{x}}) + C
    \end{displaymath}


\end{document}