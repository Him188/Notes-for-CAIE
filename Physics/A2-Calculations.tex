\documentclass[a4paper,9pt]{scrartcl}
\usepackage[nuk]{babel}
\usepackage[utf8]{inputenc}
\usepackage{amssymb,amsmath}
\usepackage{geometry}
\geometry{a4paper,left=18mm,right=18mm, top=2cm, bottom=2cm}

\title{A2 Physics}

\usepackage{lipsum}
\usepackage{ctex}
\usepackage{enumerate}
\usepackage{listings}
\usepackage{graphicx}

\begin{document}
    \section{Radioactive decay}
    w19 42 Q12

    Radon-222 is a radioactive gas. Decay constant is $7.55 \times 10^{-3} hour^{-1}$. The activity of radon gans in a sample of $4.8\times 10^{-3}m^3$ of air taken from a building is $0.6000$Bq.
    \newline There are $2.52\times 10^{25}$ are molecules in a volume of $1.00 m^3$ of air.
    \newline Calculate, for $1.00 m^3$ of the air, the ratio

    \newline

    \begin{displaymath}
        \frac{number of air molecules}{number of radon atoms}
    \end{displaymath}

    Solution:
    \begin{displaymath}
        A={\lambda}N
        N=\frac{A}{\lambda}=\frac{\frac{4.600}{4.8\times 10^{-3}}}{\frac{7.55\times 10^{-3}}{3600}} =5.96\times 10^7
    \end{displaymath}
\end{document}